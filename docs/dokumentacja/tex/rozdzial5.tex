\newpage
\section{Wnioski}

W ramach przeprowadzonego projektu udało się zrealizować pełną implementację algorytmu sortowania przez scalanie (Merge Sort),
który został zastosowany do sortowania tablic liczb całkowitych. Aplikacja została zaprojektowana zgodnie z zasadami obiektowości,
co pozwoliło na zachowanie elastyczności i łatwości w rozbudowie oraz utrzymaniu kodu. Wszystkie operacje na tablicach, takie jak dodawanie elementów,
sortowanie, tasowanie czy odwracanie, zostały zaimplementowane w sposób modularny, co ułatwia późniejsze modyfikacje oraz rozbudowę aplikacji.

Najważniejszym elementem tego projektu, poza samą implementacją algorytmu, było zastosowanie testów jednostkowych.
Testy te stanowiły kluczowy element procesu weryfikacji poprawności działania algorytmu oraz innych funkcji aplikacji.
Dzięki testom jednostkowym udało się upewnić, że algorytm działa poprawnie w różnych przypadkach granicznych,
takich jak tablice puste, posortowane, odwrotnie posortowane czy zawierające duplikaty.
Testy te są niezbędnym narzędziem w procesie zapewniania jakości oprogramowania, ponieważ pozwalają na szybkie wykrycie błędów oraz weryfikację,
że zmiany wprowadzone w kodzie nie wprowadzają nowych problemów.

\subsection{Zastosowanie testów jednostkowych}

Testy jednostkowe stanowiły fundament procesu rozwoju projektu.
Dzięki zastosowaniu frameworka Google Test, możliwe było automatyczne testowanie każdego z komponentów aplikacji, co zapewniło,
że każda funkcjonalność działa zgodnie \\ z oczekiwaniami.
Przeprowadzone testy jednostkowe obejmowały zarówno proste przypadki, jak i bardziej złożone, takie jak testowanie dużych tablic oraz tablic z duplikatami.
Użycie testów jednostkowych pozwoliło także na szybkie wykrycie problemów, które mogłyby pojawić się w przypadku modyfikacji kodu,
zapewniając tym samym większą stabilność i niezawodność aplikacji.

\newpage

\subsection{Podsumowanie wyników}

Algorytm Merge Sort, implementowany w ramach tego projektu, działa poprawnie na różnych zestawach danych wejściowych.
Został on skutecznie zaimplementowany i przetestowany, a jego działanie zostało potwierdzone przez pozytywne wyniki testów jednostkowych.
Testy wykazały, że algorytm działa zgodnie z oczekiwaniami, niezależnie od układu początkowego tablicy.
Algorytm efektywnie radzi sobie z tablicami o dużych rozmiarach oraz tymi zawierającymi duplikaty, co potwierdza jego poprawność.

\subsection{Wnioski końcowe}

Projekt ten pokazał, jak ważne jest stosowanie testów jednostkowych w procesie programowania.
Dzięki testom jednostkowym, cały projekt zyskał większą stabilność, a błędy zostały szybko zidentyfikowane i naprawione.
Przeprowadzenie testów na różnych zestawach danych umożliwiło wyciągnięcie wniosków na temat działania algorytmu w różnych scenariuszach, a także zapewniło,
że algorytm spełnia wszystkie założenia funkcjonalne.

Z punktu widzenia implementacji,
algorytm Merge Sort sprawdził się jako skuteczne narzędzie do sortowania tablic.
Jego złożoność czasowa \(\mathcal{O}(n \log n)\) okazała się odpowiednia do sortowania dużych zbiorów danych,
co potwierdziły testy na tablicach o dużej liczbie elementów. Dodatkowo, zastosowanie podejścia rekurencyjnego w połączeniu z metodą scalania zapewnia,
że algorytm działa w sposób wydajny, \\ a jego implementacja jest stosunkowo prosta do zrozumienia i utrzymania.
