\newpage
\section{Analiza problemu}
\label{sec:analiza-problemu}

\subsection{Zastosowanie algorytmu sortowania przez scalanie}

Algorytm sortowania przez scalanie (Merge Sort) jest jednym z najczęściej wykorzystywanych algorytmów do sortowania dużych zbiorów danych w informatyce. Jego główną zaletą jest stabilność i efektywność w przypadku dużych zbiorów danych. Jest to algorytm typu „dziel i zwyciężaj”, który dzieli zbiór na mniejsze podzbiory, a następnie scala je w sposób uporządkowany. Algorytm ten jest wykorzystywany \\ w różnych dziedzinach, takich jak:

\begin{itemize}
  \item Sortowanie dużych zbiorów danych w bazach danych.
  \item Przetwarzanie danych w systemach rozproszonych, gdzie duże ilości danych muszą być uporządkowane przed dalszym przetwarzaniem.
  \item Algorytmy związane z analizą danych, takie jak sortowanie wyników wyszukiwania.
  \item Problemy związane z inżynierią oprogramowania, w których dane muszą być uporządkowane w sposób stabilny (np. przy sortowaniu użytkowników po wielu kryteriach).
\end{itemize}

W niniejszym projekcie zaimplementowano algorytm Merge Sort, który jest wykorzystywany do sortowania tablic liczb całkowitych.
Program implementuje ten algorytm w języku C++, zapewniając jego elastyczność i możliwość testowania \\ w różnych scenariuszach.

\subsection{Opis algorytmu sortowania przez scalanie}

Algorytm Merge Sort działa na zasadzie podziału problemu na mniejsze podproblemy, co pozwala na efektywne sortowanie danych. Ogólny przebieg działania algorytmu przedstawia się następująco:

\begin{enumerate}
  \item Jeśli tablica ma więcej niż jeden element, dzielimy ją na dwie części (połowy).
  \item Rekursywnie sortujemy każdą z części.
  \item Po posortowaniu każdej z części, łączymy (scalamy) je w sposób uporządkowany w jedną, posortowaną tablicę.
\end{enumerate}

Przykład: Rozważmy tablicę liczb \texttt{[38, 27, 43, 3, 9, 82, 10]}. Algorytm Merge Sort wykonuje następujące kroki:

\begin{enumerate}
  \item Dzieli tablicę na dwie części: \texttt{[38, 27, 43]} i \texttt{[3, 9, 82, 10]}.
  \item Każdą z części dzieli dalej, aż do uzyskania tablic jednowymiarowych:
        \begin{itemize}
          \item \texttt{[38, 27, 43]} → \texttt{[38, 27]} i \texttt{[43]} → \texttt{[38]} i \texttt{[27]} → \texttt{[38]} i \texttt{[27]}.
          \item \texttt{[3, 9, 82, 10]} → \texttt{[3, 9]} i \texttt{[82, 10]} → \texttt{[3]} i \texttt{[9]} → \texttt{[3]} i \texttt{[9]} → \texttt{[82]} i \texttt{[10]}.
        \end{itemize}
  \item Następnie łączymy elementy w sposób uporządkowany:
        \begin{itemize}
          \item \texttt{[38]} i \texttt{[27]} → \texttt{[27, 38]}.
          \item \texttt{[38, 27]} i \texttt{[43]} → \texttt{[27, 38, 43]}.
          \item \texttt{[3]} i \texttt{[9]} → \texttt{[3, 9]}.
          \item \texttt{[82]} i \texttt{[10]} → \texttt{[10, 82]}.
          \item \texttt{[3, 9]} i \texttt{[10, 82]} → \texttt{[3, 9, 10, 82]}.
        \end{itemize}
  \item Na koniec scalamy dwie posortowane tablice:
        \begin{itemize}
          \item \texttt{[27, 38, 43]} i \texttt{[3, 9, 10, 82]} → \texttt{[3, 9, 10, 27, 38, 43, 82]}.
        \end{itemize}
\end{enumerate}

Ostateczny wynik to posortowana tablica: \texttt{[3, 9, 10, 27, 38, 43, 82]}.

\subsection{Sposób wykorzystania algorytmu w projekcie}

W projekcie algorytm Merge Sort został zaimplementowany jako część klasy \texttt{MergeSorter}, której zadaniem jest sortowanie tablicy liczb całkowitych. Użytkownik może interaktywnie dodawać, usuwać lub tasować elementy w tablicy, a następnie uruchomić algorytm, aby posortować tablicę. Proces ten jest uproszczony do kilku kroków, które opisano poniżej.

\begin{itemize}
  \item Po uruchomieniu aplikacji użytkownik ma dostęp do menu, w którym może wybrać jedną z dostępnych operacji na tablicach.
  \item Wybór opcji "Sortowanie" uruchamia algorytm Merge Sort, który wykonuje operację sortowania na aktualnej tablicy.
  \item Użytkownik może także zainicjować generowanie losowej tablicy lub odwrócić kolejność elementów w tablicy, aby przeprowadzić sortowanie w różnych scenariuszach.
  \item Po zakończeniu sortowania, wynikowa tablica zostaje wyświetlona na ekranie.
\end{itemize}

Przykład:
1. Użytkownik generuje losową tablicę: \texttt{[38, 27, 43, 3, 9, 82, 10]}.
2. Uruchamia sortowanie przez scalanie.
3. Aplikacja wyświetla posortowaną tablicę: \texttt{[3, 9, 10, 27, 38, 43, 82]}.

\subsection{Opis narzędzi i technologii użytych w projekcie}

W projekcie zastosowano szereg narzędzi i technologii, które wspierają rozwój, testowanie oraz dokumentowanie aplikacji. Do głównych narzędzi należą:

\begin{itemize}
  \item \textbf{C++}: Język programowania użyty do implementacji algorytmu oraz aplikacji. C++ zapewnia wysoką wydajność, co jest istotne przy pracy z dużymi zbiorami danych.
  \item \textbf{Google Test}: Framework do testowania jednostkowego w C++. Google Test umożliwia automatyczne uruchamianie testów oraz weryfikację poprawności działania algorytmu w różnych scenariuszach.
  \item \textbf{CMake}: System budowania, który umożliwia generowanie plików konfiguracyjnych dla różnych środowisk kompilacji. CMake zapewnia elastyczność \\ w budowaniu projektu na różnych systemach operacyjnych.
  \item \textbf{Doxygen}: Narzędzie do automatycznego generowania dokumentacji z komentarzy w kodzie. Doxygen pozwala na tworzenie dokumentacji technicznej \\ w formacie HTML lub PDF.
  \item \textbf{GitHub Actions}: Narzędzie do automatyzacji procesów CI/CD. GitHub Actions służy do automatycznego uruchamiania testów, generowania dokumentacji i wdrażania aplikacji.
\end{itemize}
